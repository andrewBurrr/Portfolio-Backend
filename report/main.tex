%! Author = andrew
%! Date = 2023-02-09

% Preamble
\documentclass[11pt]{article}

% Packages
\usepackage{main}

% Document
\begin{document}
    %! Author = andrew
%! Date = 2023-02-09

\begin{titlepage}
    \begin{center}

        \vspace*{1cm}
        \normalfont\normalsize\textsc{University Of Calgary\\ Department Of Computer Science\\ Student Portfolio}\\ [8pt] \rule{\linewidth}{0.5pt}\\ [6pt] \huge Django Interactive Project Hub\\ \rule{\linewidth}{2pt} \\ [10pt]

        \vspace{1cm}
        \normalfont\Large\textsc{Andrew M. Burton}\\
        \vspace{1cm}

        \today\\

        \vspace{8cm}

        \includegraphics[width=0.4\textwidth]{assets/images/UCalgary Logo.png}
    \end{center}
\end{titlepage}
    \tableofcontents
    \begin{abstract}
        The purpose of this project is to create a web application that allows students to showcase their projects in a professional and accessible manner. The platform will provide a comprehensive collection of mediums required for each contribution, including a video demo, blog-post style explanation, repository, and an interactive demo. The initial version of the platform will target Python applications, with support for other projects planned for future versions. The development stack for this application will make use of Django as the backend framework, PostgreSQL as the database management system, React.js as the frontend framework, and Docker to aid in the deployment process. Additionally, various other libraries and platforms will be utilized to ensure a robust and efficient user experience. The platform will be designed with security and scalability in mind, making use of Django's built-in security features and React's reusable components. The use of PostgreSQL will ensure efficient and reliable data management, allowing for quick and seamless interaction with the platform's features.
    \end{abstract}
    \section{Introduction}
    In today's rapidly changing technology landscape, it has become increasingly important for students and aspiring software developers to have a platform to showcase their projects and skills. With a plethora of projects and contributions being made every day, it can be challenging to effectively demonstrate one's work and stand out from the crowd. This project aims to address this issue by providing a web application that allows students to showcase their projects in a comprehensive and accessible manner. By providing a collection of mediums required for each contribution, including a video demo, blog-post style explanation, repository, and an interactive demo, students will be able to effectively demonstrate their projects and share their work with a wider audience. The platform will also serve as a hub for the technology community, connecting students, exchanging ideas, and building connections. This will provide students with the opportunity to receive feedback from their peers, collaborate with others, and gain exposure to potential employers and clients. In conclusion, the creation of a student project hosting platform is a critical step towards addressing the need for students to showcase their projects and skills in a professional and accessible manner. By utilizing Django, React, and PostgreSQL, the platform will ensure a secure, efficient, and user-friendly experience for all users.
    \section{Getting Started}
    Configuration:
    To run this project locally, clone this repository into a directory on your machine
    - \verb|git clone |
\end{document}